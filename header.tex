\documentclass[10pt, conference, compsoc]{IEEEtran}
%bibliography style
\usepackage[numbers]{natbib}

% input encoding
\usepackage[utf8]{inputenc}

% PDF figures
\usepackage[pdftex]{graphicx}
\DeclareGraphicsExtensions{.pdf}

% additional packages
\usepackage{url}

% set PDF version to 1.6
\pdfminorversion=6

%acronyms
\usepackage[printonlyused,withpage]{acronym}


% http://texblog.org/2011/02/26/generating-dummy-textblindtext-with-latex-for-testing/
\usepackage[english]{babel}
\usepackage{blindtext}

%listings (code blocks)
\usepackage{listings}
\usepackage{xcolor}

\definecolor{mygreen}{rgb}{0,0.6,0}
\lstset{ %
  basicstyle=\footnotesize\ttfamily,        % the size of the fonts that are used for the code
% breakatwhitespace=true,          % sets if automatic breaks should only happen at whitespace
  breakindent=1em,
  breaklines=true,                 % sets automatic line breaking
  commentstyle=\color{mygreen},  % comment style
  keepspaces=true,                 % keeps spaces in text, useful for keeping indentation of code (possibly needs columns=flexible)
  %keywordstyle=\color{blue},       % keyword style
  %linewidth=\textwidth,
  %morekeywords={*,...},            % if you want to add more keywords to the set
  %numbers=left,                    % where to put the line-numbers; possible values are (none, left, right)
  %numbersep=5pt,                   % how far the line-numbers are from the code
  %numberstyle=\tiny\color{gray},   % the style that is used for the line-numbers
  showlines=true,
  stepnumber=1,                    % the step between two line-numbers. If it's 1, each line will be numbered
  %stringstyle=\color{red},         % string literal style
  %tabsize=2,                       % sets default tabsize to 2 spaces
  title=\lstname                   % show the filename of files included with \lstinputlisting; also try caption instead of title
}
